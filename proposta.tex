\documentclass[12pt]{article}
\usepackage[utf8]{inputenc}
\usepackage[brazil]{babel}
\usepackage{sbc-template}
\usepackage{graphicx,url}
\usepackage[table,xcdraw]{xcolor}
\usepackage{longtable}

\sloppy

\title{Mestrado em Informática\\ Proposta de Trabalho \\ Extração e Analise de dados textual on line para verificar a viabilidade de startups na area de tecnologia}

\author{Ricardo Moraes Guimarães\inst{1}}


\address{Universidade Federal do Amazonas - UFAM\\
\email{ricardommoraes@gmail.com}
}

\begin{document}

\maketitle

\section{Introdução} \label{sec:intro}
O aumento do uso e popularidade da internet nos trouxe acesso livre a varios tipos de informações de produtos, pessoas, empresas, comércios entre outros. Essas mudanças transformaram a \textit{web} em um imenso repositório de informações para o mundo inteiro. Pesquisas relacionadas com geração de conhecimento a partir de informações extraídas dessas fontes públicas e acessíveis de dados são úteis, fascinantes e ao mesmo tempo desafiadoras \cite{liu:11}.

Mais recentemente houve um interesse em desenvolver técnicas de pesquisas e ferramentas para realizar pesquisas inteligentes para analise de negócios, utilizando dados textuais para gerar informações sobre negócios, marketing ou aspectos de inovação de alguns setores da industria. As pesquisa são realizadas com dados econtrados na \textit{web}, blogs, foruns ou sites de review e redes sociais. As informações podem ser colhidas também a partir de informações online de determinadas empresas \cite{Tollo:15}. 

Tais pesquisas podem ser feitas utilizando técnicas e métodos analiticos 



Tais informações podem ser utilizadas para análises de viabilidade de startups levando em consideração fatores externos e internos referente ao empreendimento.



\section{Objetivos} \label{sec:objec}

Neste trabalho, buscaremos métodos de extração e analise de dados textuais a partir das pesquisas realizadas na web e aplicadas ao principal componente análisado (PCA) para verificar os fatores de sucesso na criação de uma startup. É importante ressaltar os seguintes objetivos para realização des trabalho:

\begin{itemize}
	\item{Elaborar e desenvolver métricas para a análise e uso de informações, aplicados em startup.}
	\item{Desenvolver técnica de estração de dados textual aplicado ao princiapl componente analisado (PCA).}
	\item{Elaborar e desenvolver métricas para a análise e uso de informações, aplicados em startup.}
\end{itemize}


\section{Metodologia} \label{sec:Metod}

\section{Cronograma} \label{sec:Crono}

\begin{longtable}[c]{|l|l|l|l|l|l|l|}
\caption{Cronograma} \label{my-label}\\ \hline
\endfirsthead
\endhead
\multicolumn{1}{|c|}{\textbf{Primeiro Ano}} & \multicolumn{6}{c|}{\cellcolor[HTML]{ECF4FF}Meses} \\ \hline
\rowcolor[HTML]{ECF4FF} 
\textbf{Atividades} & 1-2 & 3-4 & 5-6 & 7-8 & 9-10 & 11-12 \\ \hline
Cursar Disciplinas & \cellcolor[HTML]{C0C0C0} & \cellcolor[HTML]{C0C0C0} & \cellcolor[HTML]{C0C0C0} & \cellcolor[HTML]{C0C0C0} & \cellcolor[HTML]{C0C0C0} & \cellcolor[HTML]{C0C0C0} \\ \hline
Levantamento Bibliográfico & \cellcolor[HTML]{C0C0C0} & & & & & \\ \hline
Estudo do Problema & & \cellcolor[HTML]{C0C0C0} & \cellcolor[HTML]{C0C0C0} & & & \\ \hline
Definição da Metodologia & & & & \cellcolor[HTML]{C0C0C0} & \cellcolor[HTML]{C0C0C0} & \\ \hline
Defesa da Proposta de Dissertação & & & & & \cellcolor[HTML]{C0C0C0} \\ \hline
\multicolumn{1}{|c|}{\textbf{Segundo Ano}}  & \multicolumn{6}{c|}{\cellcolor[HTML]{ECF4FF}Meses} \\ \hline \rowcolor[HTML]{ECF4FF} 
\textbf{Atividades} & 1-2 & 3-4 & 5-6 & 7-8 & 9-10 & 11-12 \\ \hline
Validação da Metodologia & \cellcolor[HTML]{C0C0C0} & \cellcolor[HTML]{C0C0C0} & \cellcolor[HTML]{C0C0C0} & & & \\ \hline
Elaboração da Dissertação & & & \cellcolor[HTML]{C0C0C0} & \cellcolor[HTML]{C0C0C0} & \cellcolor[HTML]{C0C0C0} & \\ \hline
Defesa da Dissertação & & & & & & \cellcolor[HTML]{C0C0C0} \\ \hline
\end{longtable}

\section{Trabalhos Relacionados} \label{sec:trab-rel}

Existem alguns trabalhos relacionados com a extração de dados textuais online, dentre eles, consideramos o trabalho de Giacomo di Tollo

Este trabalho possui semelhanças a abordagem utilizada no trabalho de Go e Bhayani, no entanto, fazemos abordagem na língua portuguesa com suas diversidades principalmente no campo semântico.

\section{Resultados Esperados} \label{sec:Result}

\section{Conclusão} \label{sec:concl}

\newpage
\bibliographystyle{sbc}
\bibliography{proposta}

\end{document}