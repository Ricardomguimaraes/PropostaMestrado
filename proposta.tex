\documentclass[12pt]{article}
\usepackage[utf8]{inputenc}
\usepackage[brazil]{babel}
\usepackage{sbc-template}
\usepackage{graphicx,url}
\usepackage[table,xcdraw]{xcolor}
\usepackage{longtable}

\sloppy

\title{Mestrado em Informática\\ Proposta de Trabalho \\ Analise de dados para verificar a viabilidade de Startups Tecnlógicas na área de tecnologia}

\author{Ricardo Moraes Guimarães\inst{1}}


\address{Universidade Federal do Amazonas - UFAM\\
\email{ricardommoraes@gmail.com}
}

\begin{document}

\maketitle

\section{Introdução} \label{sec:intro}

Atualmente, as universidades ganharam uma nova missão, a de que não é apenas formar recursos humanos na fronteira do conhecimento, mas fazer com que esses recursos sejam capazes de traduzir conhecimento na forma de novos negócios e ou produtos, processos e serviços. Além de mencionar mais uma missão das universidades, o autor \cite{hannon2013entrepreneurial:2013} ao citar \cite{gibb2012exploring:2012}, menciona que é preciso criar, dentro das universidades ambientes mais propícios para o desenvolvimento de mentalidades e comportamentos empreendedores, é importante que as próprias universidades possam pensar estratégicamente de forma mais empreendedora.
Juntamente com essa missão, estão as politicas de CT\&I no que tange a criação de novas imcubadoras e parques tecnológicos. Em conjunto, espera-se que essas políticas gerem impactos de natureza econômica, tais como aumento de oferta de empregos qualificados e de produtos com maior valor agregado e com maior potencial de competitividade internacional \cite{hannon2013entrepreneurial:2013}. Essa fenômeno tem como importância estratégica o aumento da inovação e o progresso tecnológico \cite{hannon2013entrepreneurial:2013} para Região Norte.

Segundo \cite{dimitriadis2008opinia:2008}, conforme citado por \cite{staniewski2015motivating:2015}, muitos estudos mostram que o empreendedorismos tem papel importante no desenvolvimento social e econômico. Influenciando no crescimento econômico, na criação de melhores lugares de trabalho, melhoria na produtividade da força de trabalho, contribui ainda para a criação de novas tecnologias, bens e serviços e aumenta a concorrência no mercado local.

Com intuíto de propor um corpo de conhecimento para que o empreendedor possa desenvolver sua startup com a menor impedância possível, bem como encontrar rapidamente os caminhos necessários para o desenvolvimento estrátegico do seu negócio. E acreditando que um melhor caminho é a promoção de um ambiente inovador e que seja ao mesmo tempo cheio de possibilidades, mas, e principalmente, autônomo. O presente trabalho tem como intuíto responder a uma pergunta crítica:

Como a cultura molda a percepção individual em relação às barreiras para empreender e em relação às intenções para se tornar um empreendedor acadêmico?

Modelos para avaliar e auxiliar o desenvolvimento das empresas é o principal interesse nas pesquisas, ensino e práticas comerciais. Um vasto número de manuais foram escritos que tratam de forma mais especifica de diferentes aspectos do desenvolvimento de novos negócios e que fornecem consideráveis conselhos práticos sobre com uma empresa pode aumentar suas chances de sobrevivência e sucesso \cite{davidsson2003business:2003}.

Para esta proposta, será feito uma pesquisa baseado no modelo \textbf{Businnes Platform Model} proposto por \textit{Magnus Klofsten}, a fim de avaliar a percepção de cada empreendedor em relação às barreiras para empreender e em relação às intenções para se tornar um empreendedor acadêmico. A implicação deste estudo 

Se o empreendedorismo produz tantos benefícios, parece essencial explorar os fatores que poderiam facilitar ou dificultar o início da própria atividade comercial \cite{staniewski2015motivating:2015}.

\section{Objetivos} \label{sec:objec}

Neste trabalho, buscaremos responder ao questionamento que estimulou o desenvolvimento 	deste projeto  métodos de extração e analise de dados textuais a partir das pesquisas realizadas na web e aplicadas ao principal componente análisado (PCA) para verificar os fatores de sucesso na criação de uma startup. É importante ressaltar os seguintes objetivos para realização des trabalho:

\begin{itemize}
	\item{Elaborar e desenvolver métricas para a análise e uso de informações, aplicados em startup.}
	\item{Desenvolver técnica de estração de dados textual aplicado ao princiapl componente analisado (PCA).}
	\item{Elaborar e desenvolver métricas para a análise e uso de informações, aplicados em startup.}
\end{itemize}


\section{Metodologia} \label{sec:Metod}

Esta pesquisa exploratótia 
De acordo com a proposta apresentada no livro (The Business Platform: Entrepreneurship  management in
the early stages of a firm's development"), propoe-se métricas, onde

O aumento do uso e popularidade da internet nos trouxe acesso livre a varios tipos de informações de produtos, pessoas, empresas, comércios entre outros. Essas mudanças transformaram a \textit{web} em um imenso repositório de informações para o mundo inteiro. Pesquisas relacionadas com geração de conhecimento a partir de informações extraídas dessas fontes públicas e acessíveis de dados são úteis, fascinantes e ao mesmo tempo desafiadoras \cite{liu:11}.

%Mais recentemente houve um interesse em desenvolver técnicas de pesquisas e ferramentas para realizar pesquisas inteligentes para analise de negócios, utilizando dados textuais para gerar informações sobre negócios, marketing ou aspectos de inovação de alguns setores da industria.  As informações podem ser colhidas também a partir de informações online de determinadas empresas \cite{Tollo:15}.
Tais pesquisas pesquisa serão realizadas com dados econtrados em sites das empresas imcubadas no ICOMP UFAM, no CDTEC UFAM e incubadora do IFAM e do IDSM na busca por padrões que possam ser medidos para elaboração de recomendações.

\section{Contextualização Bibliográfica} \label{sec:Contex. Bibli}


\section{Cronograma} \label{sec:Crono}

\begin{longtable}[c]{|l|l|l|l|l|l|l|}
\caption{Cronograma} \label{my-label}\\ \hline
\endfirsthead
\endhead
\multicolumn{1}{|c|}{\textbf{Primeiro Ano}} & \multicolumn{6}{c|}{\cellcolor[HTML]{ECF4FF}Meses} \\ \hline
\rowcolor[HTML]{ECF4FF} 
\textbf{Atividades} & 1-2 & 3-4 & 5-6 & 7-8 & 9-10 & 11-12 \\ \hline
Cursar Disciplinas & \cellcolor[HTML]{C0C0C0} & \cellcolor[HTML]{C0C0C0} & \cellcolor[HTML]{C0C0C0} & \cellcolor[HTML]{C0C0C0} & \cellcolor[HTML]{C0C0C0} & \cellcolor[HTML]{C0C0C0} \\ \hline
Levantamento Bibliográfico & \cellcolor[HTML]{C0C0C0} & & & & & \\ \hline
Estudo do Problema & & \cellcolor[HTML]{C0C0C0} & \cellcolor[HTML]{C0C0C0} & & & \\ \hline
Definição da Metodologia & & & & \cellcolor[HTML]{C0C0C0} & \cellcolor[HTML]{C0C0C0} & \\ \hline
Defesa da Proposta de Dissertação & & & & & \cellcolor[HTML]{C0C0C0} \\ \hline
\multicolumn{1}{|c|}{\textbf{Segundo Ano}}  & \multicolumn{6}{c|}{\cellcolor[HTML]{ECF4FF}Meses} \\ \hline \rowcolor[HTML]{ECF4FF} 
\textbf{Atividades} & 1-2 & 3-4 & 5-6 & 7-8 & 9-10 & 11-12 \\ \hline
Validação da Metodologia & \cellcolor[HTML]{C0C0C0} & \cellcolor[HTML]{C0C0C0} & \cellcolor[HTML]{C0C0C0} & & & \\ \hline
Elaboração da Dissertação & & & \cellcolor[HTML]{C0C0C0} & \cellcolor[HTML]{C0C0C0} & \cellcolor[HTML]{C0C0C0} & \\ \hline
Defesa da Dissertação & & & & & & \cellcolor[HTML]{C0C0C0} \\ \hline
\end{longtable}

\section{Trabalhos Relacionados} \label{sec:trab-rel}

Existem alguns trabalhos relacionados com a extração de dados textuais online, dentre eles, consideramos o trabalho de Giacomo di Tollo

Este trabalho possui semelhanças a abordagem utilizada no trabalho de Go e Bhayani, no entanto, fazemos abordagem na língua portuguesa com suas diversidades principalmente no campo semântico.

\section{Resultados Esperados} \label{sec:Result}

\section{Conclusão} \label{sec:concl}

\newpage
\bibliographystyle{sbc}
\bibliography{proposta}

\end{document}